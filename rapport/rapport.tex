\documentclass[12pt]{article}

\usepackage[utf8]{inputenc}
\usepackage[T1]{fontenc}
\usepackage[french]{babel}
\usepackage{hyperref}
\usepackage{lastpage}
\usepackage{graphicx}
\usepackage{listings}


\begin{document}
\begin{titlepage}
\newcommand{\HRule}{\rule{\linewidth}{0.5mm}}
\setlength{\topmargin}{0in}
\begin{minipage}{0.4\textwidth}
	\begin{flushleft} \large
		\hspace*{-0.5cm}
		\includegraphics[scale=0.20]{./ressources/logo-unicaen.jpg}
	\end{flushleft}
\end{minipage}
\center
\textsc{\large }\\[0.5cm]
\HRule \\[0.4cm]
{ \huge \bfseries Rapport Aide à la résolution de Kakuro}\\[0.4cm]
\HRule \\[1cm]
\begin{minipage}{\textwidth}
	\begin{flushright} \huge
		\center
		Compléments de POO
	\end{flushright}
\end{minipage}\\[1cm]
\textsc{\large }\\[1cm]
\begin{minipage}{\textwidth}
	\begin{flushright} \large
		\center
		PIGNARD Alexandre - 21701890\\BOCAGE Arthur - 21806332
		\\[1cm]
		L3 Informatique - Promotion 2020-2021
	\end{flushright}
\end{minipage}
\textsc{\large }\\[2cm]
{\large \today}\\[0.5cm]
\vfill
\end{titlepage}

\thispagestyle{empty}
\setcounter{page}{0}
\newpage

\tableofcontents
\newpage


\section{Introduction}
Le \textbf{Kakuro} est un jeu logique semblable aux mots croisés. Le jeu est originaire du Jpaon où sa popularité est immense. Le jeu est similaire aux mots fléchés dans lesquels une même combinaison de chiffres ne peut être utilisé deux fois dans la même grille. Bien que le jeu ne soit parvenu en Frnace que vers les années 2004 et 2005 dans le sillage du sudoku, le jeu reste connu depuis longtemps.  \\ Source : Wikipedia

\section{Utilisation du programme}
Pour utiliser notre programme, vous pouvez vous rendre dans le répertoire $Kakuro_Helper$ et lancer la commande :
\begin{lstlisting}[language=bash]
  $python3 Main.py
\end{lstlisting}


\section{Conception du programme}
Lorem ipsum dolor sit amet, consectetur adipiscing elit. Nulla eros mauris, mollis et pretium sit amet, vulputate vel magna. Curabitur ac dolor felis. Pellentesque ut turpis posuere, consequat est eu, eleifend turpis. Maecenas tincidunt, felis at volutpat ornare, turpis dolor dapibus arcu, vitae ullamcorper erat lacus ut lectus. Pellentesque ornare quis tellus ac dapibus. Nunc congue mollis lacus in hendrerit. Maecenas justo sem, gravida mattis nulla a, ornare porttitor quam. Cras molestie, nulla convallis congue cursus, mi ex blandit turpis, non pharetra arcu quam at mauris. Nullam velit massa, luctus eget elit ut, ultricies volutpat sapien. Quisque blandit sem felis, non varius enim blandit quis.
%\begin{figure}[ht]
%  \begin{center}
%    \includegraphics[width=\textwidth]{./ressources/taquin.png} 
%  \end{center}
%  \caption{Diagrammes de packages du programme}
%\end{figure}

Pellentesque ornare quis tellus ac dapibus. Nunc congue mollis lacus in hendrerit. Maecenas justo sem, gravida mattis nulla a, ornare porttitor quam. Cras molestie, nulla convallis congue cursus, mi ex blandit turpis, non pharetra arcu quam at mauris. Nullam velit massa, luctus eget elit ut, ultricies volutpat sapien.
\subsection{Modèle}


\end{document}